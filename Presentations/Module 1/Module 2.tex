\documentclass{beamer}
\usetheme{metropolis}           % Use metropolis theme
\title{Python for Data Science}
\date{\today}
\author{Franco Calle}
\institute{Princeton University}
\begin{document}
  \maketitle
  \section{Introduction}
	\begin{frame}{Why is it important to learn a programming language?}
		\begin{itemize}
		\item No obstante los avances de las últimas décadas, el trásito de educación básica a superior sigue siendo un desafío en América Latina.
		\item Existe evidencia de que proveer información sobre los retornos a la educación informa a los estudiantes, mejorando sus logros educativos en una forma  \textbf{costo-efectiva} (Nguyen 2008, Jensen 2010, Berry et al. 2017, Neilson et. al. 2017). 
		\end{itemize}
	\end{frame}
	\begin{frame}{Learning curve and scope of programming languages (R, Stata, C+, Python)}
		\begin{itemize}
		\item No obstante los avances de las últimas décadas, el trásito de educación básica a superior sigue siendo un desafío en América Latina.
		\item Existe evidencia de que proveer información sobre los retornos a la educación informa a los estudiantes, mejorando sus logros educativos en una forma  \textbf{costo-efectiva} (Nguyen 2008, Jensen 2010, Berry et al. 2017, Neilson et. al. 2017). 
		\end{itemize}
	\end{frame}

	\begin{frame}{What is Data Science? What is Machine Learning?}
		\begin{itemize}
		\item No obstante los avances de las últimas décadas, el trásito de educación básica a superior sigue siendo un desafío en América Latina.
		\item Existe evidencia de que proveer información sobre los retornos a la educación informa a los estudiantes, mejorando sus logros educativos en una forma  \textbf{costo-efectiva} (Nguyen 2008, Jensen 2010, Berry et al. 2017, Neilson et. al. 2017). 
		\end{itemize}
	\end{frame}

	\begin{frame}{Why Python?}
		\begin{itemize}
		\item No obstante los avances de las últimas décadas, el trásito de educación básica a superior sigue siendo un desafío en América Latina.
		\item Existe evidencia de que proveer información sobre los retornos a la educación informa a los estudiantes, mejorando sus logros educativos en una forma  \textbf{costo-efectiva} (Nguyen 2008, Jensen 2010, Berry et al. 2017, Neilson et. al. 2017). 
		\end{itemize}
	\end{frame}

  \section{Python Basics}



\end{document}