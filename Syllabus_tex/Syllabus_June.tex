
\documentclass[12pt]{article}
\usepackage[margin=1in]{geometry}% Change the margins here if you wish.
\setlength{\parindent}{0pt} % This is the set the indent length for new paragraphs, change if you want.
\setlength{\parskip}{5pt} % This sets the distance between paragraphs, which will be used anytime you have a blank line in your LaTeX code.
\pagenumbering{gobble}% This means the page will not be numbered. You can comment it out if you like page numbers.

\usepackage{amsmath,amsthm,amssymb}

\usepackage{graphicx}
\usepackage{float}
%\usepackage{biblatex}
%\bibliographystyle{}
%\bibliography{syl_references.bib}

\usepackage{pdfpages}
\usepackage{graphicx}
\usepackage{hyperref}
\definecolor{brightmaroon}{rgb}{0.76, 0.13, 0.28}
\hypersetup{
    colorlinks=true,
    linkcolor=brightmaroon,
    filecolor=brightmaroon,
    urlcolor=brightmaroon,
}
\usepackage{fontawesome}
\usepackage[T1]{fontenc}
\usepackage{helvet}


\title{Python Aplicado a la Ciencia de Datos}


\author{\href{http://francocalle.github.io/}{Franco Calle}}

\date{}

\begin{document}

\maketitle

\section*{Resumen y Objetivos}

Python es un lenguaje de programación cuya versatilidad para el procesamiento de datos y capacidad para el análisis funcional lo hace cada vez más popular. Su popularidad ha crecido tanto en los últimos años que en el sector privado y la academia cada vez lo utilizan más para lograr distintos objetivos relacionados a la ciencia de datos como la limpieza y procesamiento de datos, automatización de procesos, hasta predicción y modelamiento del comportamiento humano.

El objetivo del curso es brindar al estudiante los fundamentos y lógica de programación en Python que sirvan como herramienta para abordar y resolver problemas que podrían ser aplicados en el sector privado y en la academia. Especificamente, al finalizar el curso, el estudiante podrá reconocer y aplicar de manera conjunta objetos como listas, dataframes, diccionarios, tuplas y funciones para resolver un problema de predicción y validarlo en la vida real.

\section*{Introducción}

\begin{itemize}
\item[-] ¿Por qué es importane aprender un lenguaje de programación?
\item[-] Curva de aprendizaje y alcance de distintos lenguajes de programación  (R, Stata, C+, Python)
\item[-] ¿Qué es Data Science? ¿Qué es Machine Learning?
\item[-] ¿Por qué Python?
\end{itemize}

\section*{Primer Módulo: Elementos Básicos de Python}

\subsection*{Primera Clase: Variables, expresiones, y statements}
\begin{itemize}
\item[-]	Values, variable names, and keywords
\item[-]	Operators, operands, expressions, the order of operations, string operations
\item[-]	Asking for inputs to the user
\item[-]	Mnemonic variable names
\end{itemize}

\subsection*{Segunda Clase: Objetos de Python}
\begin{itemize}
\item[-]	Lists
\item[-]	Dictionaries
\item[-]	Tuples
\item[-]	Multiple assignments with dictionaries
\end{itemize}

\subsection*{Tercera Clase: Ejecución Condicional}
\begin{itemize}
\item[-]	Boolean Expressions
\item[-]	Logical operators
\item[-]	Conditional, alternative, and chained conditional executions
\item[-]	Nested conditionals
\item[-]	Guardians: catching expressions using try and except
\item[-]	Short-circuit evaluation of logical expressions
\end{itemize}

\subsection*{Cuarta Clase: Iteraciones}
\begin{itemize}
\item[-]	Updating variables
\item[-]	Definite loops using for
\item[-]	Double, multiple and nested iteration
\item[-]	While statement
\item[-]	List comprehension
\end{itemize}

\subsection*{Quinta Clase: Funciones}
\begin{itemize}
\item[-]	The flow of execution, arguments, and parameters
\item[-]    Adding new functions
\item[-]	Definitions and uses
\item[-]	Annonimous functions Lambda
\item[-]	Mapping and filtering
\item[-]	Numpy module
\item[-]	Pandas module
\end{itemize}

\subsection*{Sexta Clase: DataFrames}
\begin{itemize}
\item[-]	Using and creating Dataframes
\item[-]	Replace and rename columns
\item[-]	Slicing Dataframes
\item[-]	Merge, Append
\item[-]	Import, Export
\end{itemize}

\section*{Segundo Módulo: El problema de predicción y el uso de Machine Learning}

\subsection*{Primera Clase: Predicción y Métricas de Performance}
\begin{itemize}
\item[-]	The Prediction Problem
\item[-]	Confusion Matrix
\item[-]	Accuracy, Precision, Recall
\item[-]	The receiving operations curve and Area Under the Curve

\end{itemize}

\subsection*{Segunda Clase: Métodos de Machine Learning para Clasificación}
\begin{itemize}
\item[-]	Logistic Regression
\item[-]	K-Nearest Neighbors Classification
\item[-]	Supervised Machine Learning
\item[-]	Multi – Class Classification
\item[-]	Decision Trees
\item[-]	Random Forests
\end{itemize}

\subsection*{Tercera Clase: Métodos de Machine Learning para Regressión}
\begin{itemize}
\item[-]	Least Squares
\item[-]	Ridge
\item[-]	Polynomial Regression
\item[-]	R-squared, MSE
\end{itemize}

\subsection*{Cuarta Clase: Validación Cruzada y Composición del Modelo}
\begin{itemize}
\item[-]	Cross Validation
\item[-]	Grid Search
\item[-]	Contribución de la data al modelo
\item[-]	Contribución de las variables al modelo
\end{itemize}

\section*{Proyecto Final}

El proyecto final requiere utilizar todos los conocimientos aprendidos en el Módulo 1 y 2 del curso para resolver un problema de predicción aplicado a las ciencias sociales utilizando data del INEI.

\end{document}
