
\documentclass[12pt]{article}
\usepackage[margin=1in]{geometry}% Change the margins here if you wish.
\setlength{\parindent}{0pt} % This is the set the indent length for new paragraphs, change if you want.
\setlength{\parskip}{5pt} % This sets the distance between paragraphs, which will be used anytime you have a blank line in your LaTeX code.
\pagenumbering{gobble}% This means the page will not be numbered. You can comment it out if you like page numbers.

\usepackage{amsmath,amsthm,amssymb}

\usepackage{graphicx}
\usepackage{float}
\usepackage{hyperref}


\title{Machine Learning with Python}


\author{\href{http://users.nber.org/~callef/}{Franco Calle}}

\date{}

\begin{document}

\maketitle


\section*{Summary}
Machine learning is an application of artificial intelligence that provides systems the ability to automatically learn and improve from experience. The tools of ML are increasingly adopted by academia and companies all over the world because of its predictive power in the era of big data.

For instance, academia is using ML tools to predict behavioral patterns, estimate unbiased reduced form parameters, eliminate bias in data collection and to automate research processes, among other uses. Companies are also adopting AI technologies and therefore demand professionals with high knowledge in ML that could build predictive models of consumer behavior, store sales, stock prices, among others.

The economist of the 21st century must have a strong background in Machine Learning to exploit the most valuable resource in the world: Data\footnote{The Economist: https://www.economist.com/news/leaders/21721656-data-economy-demands-new-approach-antitrust-rules-worlds-most-valuable-resource}. Big windows of opportunities are frequently opened for professionals with a strong background in ML, and one should be prepared to cross them.

\section*{Objectives}

The course aims to provide the necessary knowledge and tools of programming in python language. The student after the course will easily recognize and handle primary python objects such as lists, dictionaries, tuples, and functions.
The student will comprehend diverse methods used in machine learning and know how these methods learn from the world to estimate more accurate predictions.
The student will know how to build machine learning algorithms.

\section*{Introduction}
  
\begin{itemize}
\item[-] Why is it important to learn a programing language?
\item[-] Learning curve and scope of programming languages (R, Stata, C+, Python)
\item[-] What is Data Science? What is Machine Learning?
\item[-] Why Python?
\end{itemize}

\section*{First Module: Python Basics}

\subsection*{First Lecture: Variables, expressions, and statements}
\begin{itemize}
\item[-]	Values, variable names, and keywords
\item[-]	Operators, operands, expressions, the order of operations, string operations
\item[-]	Asking the user for input
\item[-]	Choosing mnemonic variable names
\end{itemize}

\subsection*{Second Lecture: Conditional execution}
\begin{itemize}
\item[-]	Boolean expressions
\item[-]	Logical operators
\item[-]	Conditional, alternative, and chained conditional executions
\item[-]	Nested conditionals
\item[-]	Guardians: catching expressions using try and except
\item[-]	Short-circuit evaluation of logical expressions
\end{itemize}

\subsection*{Third Lecture: Functions}
\begin{itemize}
\item[-]	Numpy module
\item[-]	Pandas module
\item[-]   Adding new functions
\item[-]	Definitions and uses
\item[-]	The flow of execution, arguments, and parameters
\end{itemize}

\subsection*{Fourth Lecture: Iteration}
\begin{itemize}
\item[-]	Updating variables
\item[-]	while statement
\item[-]	Definite loops using for
\item[-]	Loop patterns
\end{itemize}

\subsection*{Fifth Lecture: Type of objects}
\begin{itemize}
\item[-]	Lists
\item[-]	Dictionaries
\item[-]	Tuples
\item[-]	Multiple assignments with dictionaries

\end{itemize}

\section*{Second Module: Machine Learning and Artificial Intelligence}

\subsection*{First Lecture: Machine Learning Methods using scikit learn}
\begin{itemize}
\item[-]	K-Nearest Neighbors Classification
\item[-]	Supervised Machine Learning
\item[-]	Least Squares
\item[-]	Ridge, Lasso, and Polynomial Regression
\item[-]	Logistic Regression
\item[-]	Support Vector Machines
\item[-]	Multi – Class Classification
\item[-]	Cross – Validation
\item[-]	Decision Trees
\end{itemize}

\subsection*{Second Lecture: Deep Learning and Neural Networks (Intro)}
\begin{itemize}
\item[-]	Neural Networks
\item[-]	Deep Learning
\end{itemize}

\newpage

\begin{thebibliography}{9}
\bibitem{latexcompanion} 
Michel Goossens, Frank Mittelbach, and Alexander Samarin. 
\textit{The \LaTeX\ Companion}. 
Addison-Wesley, Reading, Massachusetts, 1993.
 
\bibitem{einstein} 
Albert Einstein. 
\textit{Zur Elektrodynamik bewegter K{\"o}rper}. (German) 
[\textit{On the electrodynamics of moving bodies}]. 
Annalen der Physik, 322(10):891–921, 1905.
 
\bibitem{knuthwebsite} 
Knuth: Computers and Typesetting,
\\\texttt{http://www-cs-faculty.stanford.edu/\~{}uno/abcde.html}
\end{thebibliography}



\end{document}